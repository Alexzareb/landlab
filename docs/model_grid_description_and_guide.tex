
\documentclass[12pt]{amsart}
\usepackage{geometry} % see geometry.pdf on how to lay out the page. There's lots.
%\geometry{a4paper} % or letter or a5paper or ... etc
% \geometry{landscape} % rotated page geometry

% See the ``Article customise'' template for come common customisations

\title{The ModelGrid Package}
\author{Greg Tucker}
\date{First version, May 2013} % delete this line to display the current date

%%% BEGIN DOCUMENT
\begin{document}

\maketitle
%\tableofcontents

\section{Overview}

ModelGrid is an open-source software package that creates and manages a regular or irregular grid for building 2D numerical simulation models. ModelGrid is especially useful for finite-volume (FV) and finite-difference (FD) models, but also can be used for a variety of other applications. ModelGrid provides efficient built-in functions for common operations in FD and FV models, such as calculating local gradients and integrating fluxes around the perimeter of grid cells. Staggered-grid models are especially easy to implement with ModelGrid.
A novel feature of ModelGrid is the ability to switch seamlessly between structured and unstructured grids.

This document provides a basic introduction to building applications using ModelGrid. It covers: (1) how grids are represented, (2) a tutorial example in building a diffusion-model application, and (3) a guide to ModelGrid's methods and data structures. ModelGrid is written in python. It is a component of the Landlab modeling package.

\subsection{How a Grid is Represented}




\end{document}